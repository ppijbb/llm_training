\subsection{Summary and Comparison}

We compare SPECTRA with representative MoE routing mechanisms, including both foundational methods and state-of-the-art approaches from 2024-2025.

\begin{table}[t]
\centering
\small
\begin{tabular}{lcccc}
\toprule
Method & Sequential Context & Orthogonality & Load Balancing & Complexity \\
\midrule
Switch~\cite{fedus2021switch} & $\times$ & None & Auxiliary Loss (Static) & $O(d \cdot E)$ \\
Expert Choice~\cite{zhou2022expert} & $\times$ & None & Implicit & $O(d \cdot E)$ \\
OMoE~\cite{omoe2025} & $\times$ & Auxiliary Loss & Auxiliary Loss & $O(d \cdot E)$ \\
RoMA~\cite{roma2025} & $\times$ & Manifold Align & Auxiliary Loss & $O(d \cdot E)$ \\
\textbf{SPECTRA (Ours)} & \textbf{$\checkmark$} (GRU) & \textbf{Structural} (OSR) & \textbf{Optimal Transport} & \textbf{$O(d \cdot E + T \cdot N \cdot E)$} \\
\bottomrule
\end{tabular}
\caption{Comparison of routing mechanisms. SPECTRA uniquely combines sequential context, structural orthogonality via repulsive cost, and pure optimal transport (Sinkhorn) for balancing.}
\label{tab:comparison}
\end{table}

\textbf{Key Differences:}
\begin{enumerate}
    \item \textbf{Structural Orthogonality:} Unlike OMoE/RoMA which rely on competing auxiliary losses, SPECTRA enforces separation directly in the routing transport cost via repulsion.
    \item \textbf{Context Awareness:} SPECTRA is unique in using sequential state (GRU) to maintain routing consistency across the sequence.
    \item \textbf{Parameter-free Balancing:} OSR achieves balance via Sinkhorn normalization without tuning complex load-balance loss coefficients.
\end{enumerate}

This combination of features allows SPECTRA to overcome the limitations of prior work, leading to more robust, specialized, and balanced expert utilization, which we will demonstrate in the following sections.
